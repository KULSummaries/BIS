\section{IT Doesn't Matter --- Nicholas G. Carr}
Technology has become the backbone of commerce. Companies view it as ever more critical to their success, resulting in huge spending on IT. Companies are even trying to gain a competitive advantage through the use of IT.

The idea is that IT's potency has increased, and thus its strategic value. This is wrong. A truly strategic resource is scarce. You gain an advantage over competitors with something they don't have or can't do. IT however is cheaper than ever and available to everyone. IT has become a cost of production. It's a commodity input.

A distinction needs to be made between proprietary technology and infrastructural technologies. As long as the first one remains protected, they can be used for long-term strategic advantages.

Infrastructural technologies may only provide short-term strategical advantages. Just like a railroad firm building railroad lines. At first the firm might have an andvantage by being the only one using it. In the long run, it's better for the economy at large to let other competitors use the lines as well.

Also a superior insight in how to use technology or seing how technology will change the market, can lead to an advantage.

This makes executives assume that the opportunities for advantages is indefinite. This is wrong. In the long term universal standards will take over and proprietary systems will become obsolete.

Technology still plays a role in gaining an advantage, but only on macroeconomical level. E.g. a country not building a railroad network would lag behind.

IT seems to be a infrastructural technology, accelerated by the arrival of the internet. It's a perfect delivery channel for generic web services, just like companies would buy power.

``So what should companies do? When a resource becomes essential to competition but inconsequential to strategy, the risks it creates become more important than the advantages it provides''. Thus they should identify and temper their vulnerabilities.

In the long run they should not overspend on IT. Invest in what is necessary to be competitive, but not in things that are counterproductive.
